\documentclass[a4paper,12pt]{article}
%\usepackage{amsthm}
%\theoremstyle{plain}
\newtheorem{thm}{Theorem}[section]
\newtheorem{corollary}{Corollary}[thm]
\newtheorem{prf}{Proof}[thm]
\newtheorem{defn}{Definition}[section]
\newtheorem{exmp}{Example}
%\setcounter{secnumdepth}{3}
\usepackage{siunitx}
\usepackage{microtype}
\usepackage{float}
\usepackage{graphicx}
\usepackage{bm}
\usepackage{commath}
\usepackage{amsmath}
\usepackage{parskip}
\usepackage{longtable}
\usepackage{enumerate}
\graphicspath{}
\usepackage[margin=1in]{geometry}
\usepackage{amssymb}
\usepackage{titlesec}
\usepackage{hyperref}
\usepackage{cleveref}
\linespread{1.3}
\usepackage{listings}
\usepackage{xcolor} % for setting colors

% set the default code style
\lstset{
  language=C++,
  frame=tb, % draw a frame at the top and bottom of the code block
  tabsize=4, % tab space width
  showstringspaces=false, % don't mark spaces in strings
    %numbers=left, % display line numbers on the left
  commentstyle=\color{green}, % comment color
  keywordstyle=\color{blue}, % keyword color
  stringstyle=\color{red}, % string color
  basicstyle=\ttfamily\lst@ifdisplaystyle\footnotesize\fi,
  basewidth = {.48em}
}
\newcommand*{\diff}{\mathop{}\!\mathbf{d}}
\newcommand*{\hlinebreak}{\rule{\textwidth}{1pt}}
\makeatletter
\def\thm@space@setup{%
  \thm@preskip=\parskip \thm@postskip=0pt
}
\makeatother
\title{Professional C++ notes}
\date{\today}
\author{Anas Syed}

\begin{document}
\maketitle
\tableofcontents
\newcommand{\sectionbreak}{\clearpage}

\section{Basics}
\subsection{Preprocessor directives}

\lstinline|#pragma| is not standard across all compilers, so don't use it.

\hlinebreak
\begin{lstlisting}[caption={Prevent circular includes}]
#ifndef [key]
#define [key]
//code
#endif
\end{lstlisting}

\subsection{Casting}
\begin{enumerate}
  \item \lstinline|bool someBool = (bool)someInt;|
  \item \lstinline|bool someBool = bool(someInt);|
  \item \label{itm:static_cast} \lstinline|bool someBool = static_cast<bool>(someInt);|
\end{enumerate}
\Cref{itm:static_cast} is considered to be the cleanest.

\subsection{Structs}
Structs are the same as classes in \verb|C++|, except the default access specifier for a struct is public, whereas for a class it's private. If we define a struct as follows:

\begin{lstlisting}
struct employee {
  int age;
  float salary;
  char initial;
} employee1, employee2;
\end{lstlisting}

then when we call the struct in \verb|C|, we must prefix \verb|employee| with \verb|struct|. In \verb|C++|, this is optional. Alternatively, in \verb|C|, we can \verb|typedef| the struct:

\begin{lstlisting}
typedef struct {
  int age;
  float salary;
  char initial;
} employee;
employee employee1, employee2;
\end{lstlisting}

or

\begin{lstlisting}
struct employee {
  int age;
  float salary;
  char initial;
} employee1, employee2;
typedef struct employee employee;
\end{lstlisting}
\subsection{Stack and heap}
The stack is a Last in First out data structure. If you call a function \lstinline|foo()|, then all of the static variables (those not created using \lstinline|new| or \lstinline|malloc|) in \lstinline|foo| exist in a stack frame. If \lstinline|foo| was called from \lstinline|main|, then you cannot easily change or access the static variables in the stack frame of \lstinline|foo| from within \lstinline|main|, because they are in a different stack frame.

However, if you allocate some dynamic memory to a variable in \lstinline|foo|, then you could access or modify this variable in main.
\subsubsection{Freeing memory}
When you allocate memory with the \lstinline|new| operator, you must eventually free it with \lstinline|delete|. Note that this even applies to \lstinline|int| objects. For example, the following code has a 4 byte memory leak:

\begin{lstlisting}[caption={Memory leak}]
int main() {
  int* a = new int(5);
}
\end{lstlisting}

Running \lstinline|valgrind --tool=memcheck ./a.out| gives the error message:
\begin{lstlisting}[language=bash, caption={Valgrind output}]
==11934== Memcheck, a memory error detector
==11934== Copyright (C) 2002-2013, and GNU GPL'd, by Julian Seward et al.
==11934== Using Valgrind-3.10.1 and LibVEX; rerun with -h for copyright info
==11934== Command: ./a.out
==11934== 
==11934== 
==11934== HEAP SUMMARY:
==11934==     in use at exit: 4 bytes in 1 blocks
==11934==   total heap usage: 1 allocs, 0 frees, 4 bytes allocated
==11934== 
==11934== LEAK SUMMARY:
==11934==    definitely lost: 4 bytes in 1 blocks
==11934==    indirectly lost: 0 bytes in 0 blocks
==11934==      possibly lost: 0 bytes in 0 blocks
==11934==    still reachable: 0 bytes in 0 blocks
==11934==         suppressed: 0 bytes in 0 blocks
==11934== Rerun with --leak-check=full to see details of leaked memory
==11934== 
==11934== For counts of detected and suppressed errors, rerun with: -v
==11934== ERROR SUMMARY: 0 errors from 0 contexts (suppressed: 0 from 0)
\end{lstlisting}
See \url{http://www.cprogramming.com/debugging/valgrind.html} for more information on Valgrind.

On the other hand, the following does not create a leak:
\begin{lstlisting}[caption={No memory leak}]
int main() {
  int a = int(5);
}
\end{lstlisting}

\subsection{Strings}
\subsubsection{C style}
\begin{lstlisting}
char arrayString[20] = "Hello, World"; //allocates 20 bytes
char* pointerString = "Hello, World"; //allocates 13 bytes
\end{lstlisting}

Use \lstinline|#include <cstring>| for standard \verb|C| library functions.
\subsubsection{C++ style}
\begin{lstlisting}
#include <string>
int main() {
  std::string str1 = "Hello";
  std::string str2 = str1 + ", World";
}
\end{lstlisting}
\subsection{Exceptions}
\begin{lstlisting}
#include <iostream>
#include <stdexcept>

double divideNumbers(double inNumerator, double inDenominator) {
  if (inDenominator == 0) {
    throw std::exception();
  }
  return (inNumerator / inDenominator);
}

int main(int argc, char** argv) {
  try {
    std::cout << divideNumbers(2.3, 0) << std::endl;
  } catch (std::exception exception) {
    std::cout << "An exception was caught!" << std::endl;
  }
}
\end{lstlisting}

\section{Object oriented programming}
\subsection{Access specifiers}
Any method of the class can call a protected method and access a protected member. Methods of a subclass can call a protected method or access a protected member of an object.
\subsection{Reminder of syntax}
\begin{lstlisting}[caption={SpreadsheetCell.h}]
class SpreadsheetCell {
  public:
    SpreadsheetCell();
    ~SpreadsheetCell();
    double getValue();
  private:
    double mValue;
};
\end{lstlisting}
\begin{lstlisting}[caption={SpreadsheetCell.cpp}]
#include "SpreadsheetCell.h"
SpreadsheetCell::SpreadsheetCell() : mValue(0) {
  std::cout << "Creating cell" << std::endl;
}

SpreadsheetCell::~SpreadsheetCell() {
  std::cout << "Destroying cell " << mValue << std::endl;
}

double SpreadsheetCell::getValue() {
  return (mValue);
}
\end{lstlisting}
\subsection{Creating objects on the stack or heap}
\begin{lstlisting}[caption={Creating objects on the stack}]
SpreadsheetCell myCell, anotherCell;
\end{lstlisting}
\begin{lstlisting}[caption={Creating objects on the heap}]
SpreadsheetCell* myCellp = new SpreadsheetCell();
//run some code
delete myCellp; //Don't forget to deallocate memory
\end{lstlisting}

\end{document}
